\documentclass[11pt]{article}
\usepackage{coling2014}
%\usepackage{acl2013}
\usepackage[utf8]{inputenc}
%\usepackage{natbib}
\usepackage{times}
%\usepackage{url}
\usepackage{color}
\usepackage{multirow}
\usepackage{latexsym}
\usepackage{booktabs,amsmath,multicol}
\usepackage[all]{xy}
\usepackage{dblfloatfix}
\usepackage{tikz}
%\usepackage{caption}
%\captionsetup{justification=centering}
%\setlength\titlebox{6.5cm}    % You can expand the title box if you
% really have to
% \textheight 685pt 
% \hyphenation{Buda-pest}
% \addtocounter{section}{-1}

%\DeclareRobustCommand{\rtextsuperscript}{\textsuperscript}
%\DeclareRobustCommand{\rhspace}{\hspace}

\title{Finite-state modeling of general vocabulary}

\author{Dávid Márk Nemeskey \\
HAS Computer and Automation Research Institute, \\
Kende u 13-17, \\
H-1111 Budapest \\
{\tt nemeskey.david@sztaki.mta.hu} \\\And
Francis M. Tyers \\
Dept. Lleng. i Sist. Inform., \\
Universitat d'Alacant, \\
E-03080 Alacanti \\
{\tt ftyers@dlsi.ua.es} \\}

\date{}

%%%%%%%%%%%%%%%%%%%%%%%%%%%%%%%%%%%%%%%%%%%%
% Submit to:
%   - AFL14    (http://www.inf.u-szeged.hu/afl2014/) -- Mar 14
%   - CIAA2014 (http://www.informatik.uni-giessen.de/ciaa2014/) -- Mar 9
%   - CICLing 15th (http://www.cicling.org/2014/) -- Dec 31 / Jan 7
%   - FSMNLP 2014 (?) -- ??
%
%   Nothing on EACL this year :(
%%%%%%%%%%%%%%%%%%%%%%%%%%%%%%%%%%%%%%%%%%%

\begin{document}
\maketitle

\begin{abstract}
  % TODO Add Yli-Jyra:2011
  In recent years, the problem of finite state Constraint Grammar~(CG) parsing has
  received renewed attention. Both \cite{Hulden:2011} and
  \cite{Peltonen:2011} presented compilers that convert CG rules to finite
  state transducers. While these formalisms serve their purpose as proofs of the
  concept, their performance lags behind other implementations.  %TODO implementionS?
  % TODO I am not talking about compilation, but application
  
  In this paper, we argue that fault lies with using generic finite state
  libraries, and not with the formalisms themselves. We present an open-source
  implementation that capitalizes on the characteristics of CG rule
  application to improve execution time. The implementation achieves performance
  comparable to the current open-source state of the art.
  % TODO OMG this is horrible
\end{abstract}

% TODO: Mans's quadratic claim stands only if bimachine factorization had been
%       performed!
% TODO: Bimanchine factorization, measurements

% TODO: describe Mans's method
% TODO: Apertium stream format, wchar_t vs. char

\section{Introduction}

Constraint Grammar (CG), described originally by Karlsson~\cite{Karlsson:1990},
is a rule-based formalism for various linguistics tasks, including morphological
analysis, clause boundary detection and surface syntactic parsing. It has
been used in a wide range of application areas, such as morphological
disambiguation, grammar checking and machine translation~\cite{Bick:2011}. CG
owns its popularity to two reasons: first, it achieves high accuracy on free
text. Second, it works for languages where the annotated corpora required by
statistical parsing methods are not available, but a linguist willing to work on
the rules is. The original CG has since been superseded by CG-2, due to
Tapanainen~\cite{Tapanainen:1996} and lately, the open source VISL
CG-3~\cite{Bick:2000,Didriksen:2011}.

Constrain Grammar, however, has its drawbacks, one of which is speed. The
Apertium machine translation project~\cite{Apertium:2011} uses both CG (via
VISL CG-3) and $n$-gram based models for morphological disambiguation, and while
CG achieves higher accuracy, the $n$-gram model runs about ten times faster.

In this paper, we investigate how using finite-state transducers (FST) for CG
application can help bridging the performance gap. In recent years, several
methods have been proposed for compiling a CG to FST and applying it on text:
Hulden~\cite{Hulden:2011} compiles CG rules to transducers and runs them on the
input sentences; Peltonen~\cite{Peltonen:2011} converts the sentences into ambiguous
automata and attempts to eliminate branches by intersecting them with the rule
FSTs; finally, Yli-Jyra~\cite{Yli-Jyra:2011} creates a single FST from the
grammar and applies it on featurized input. Unfortunately, none of the authors
report exact performance measurements of their systems. Yli-Jyra published
promising numbers for the preprocessing step, but nothing on the overall
performance. Peltonen, on the other hand, observed that "VISL CG-3 was 1 500
times faster" then her implementation~\cite{Peltonen:2011}.

We do not attempt here to add a new method to this list; instead, we
concentrate on three practical aspects of FST-based CG. First, we report accurate
measurements of the real-world performance of one of the methods above. Second,
we endeavor to optimize the implementation of the selected method. All three
works used \emph{foma}, an open source FST library due to Mans
Hulden~\cite{Hulden:2009a,Hulden:2009b}. We show that while \emph{foma} is fast,
relying on specialized FST application code instead of a generic library
clearly benefits performance. We also demonstrate what further improvements can
be achieved by exploiting the peculiarities of CG. Lastly, our research also
aims to fill the niche left by the lack of openly accessible finite state CG
implementations.

Sect.~\ref{sec:fomacg} briefly introduces the method we chose to evaluate.
In the rest of the paper, we present our optimizations in a way that mirrors the
actual development process. We start out with a simple rule engine based on
foma, and improve it step-by-step, benchmarking its performance after each
modification, instead of a single evaluation chapter. We start in
Sect.~\ref{sec:methodology} by describing our evaluation methodology.
Section~\ref{sec:speed} follows the evolution of the rule engine, as it improves
in terms of speed. Section~\ref{sec:complex} contains a complexity analysis
and introduces an idea that theoretically allows us to improve the average- and
best-case asymptotic bound.
Section~\ref{sec:memory} demonstrates how memory savings can be derived from the
steps taken in section~\ref{sec:speed}. Finally, Sect.~\ref{sec:conclusion}
contains our conclusions and lists the problems that remain for future work.

\section{The \emph{fomacg} compiler and \emph{fomacg-proc}}
\label{sec:fomacg}

We have chosen Mans Hulden's \emph{fomacg} compiler for our study. Our
reasons for this are twofold. The transducers generated by \emph{fomacg} were
meant to be run on the input directly, but they could also be applied to a
finite-state automaton (FSA) representation of the input sentence via FST
composition, thereby giving us more space to experiment. Peltonen's method,
on the other hand, works only through FST intersection. More importantly,
\emph{fomacg} was the only compiler that is openly, (if not obviously,)
available\footnote{In the Apertium software repository:
https://svn.code.sf.net/p/apertium/svn/branches/fomacg}.

Here we briefly describe how \emph{fomacg} works; for more details refer to
\cite{Hulden:2011}. A CG used for morphological disambiguation takes as input
a morphologically analyzed text, which consists of \emph{cohorts}: a word with
its possible meanings. A meaning (or \emph{reading} in CG parlance) is
represented by a set of morphosyntactic \emph{tags}. For example, the cohort of
the ambiguous Hungarian word \textit{szív} with two readings "heart" and
"to suck" would be
\texttt{\^{}szív/szív<n><sg><nom>/szív<vblex><pres><s3p>\$}\footnote{
The example is in the Apertium stream format, not in CG-2 style.}. The text is
tokenized into sentences based on a set of \emph{delimiters}. CG rules
operate on a sentence, removing readings from cohorts based on their context.
The rules can be divided into priority levels called \emph{section}. Most
implementations apply the rules one-by-one in a loop, until no rules can further
modify the sentence.

\emph{fomacg} expects cohorts to be encoded in a different format; the cohort in
the example above would be represented as
\begin{verbatim}
$0$ "<szív>" #BOC#        |
#0# "szív" n sg nom       |
#0# "szív" vblex pres s3p | #EOC#
\end{verbatim}
The rule transducers mark readings for removal by replacing the \texttt{\#0\#}
in front of the reading by \texttt{\#X\#}; they act as identity for sentences
they cannot be applied to.

\emph{fomacg} is only a compiler, which reads a CG rule file and emits a
\emph{foma} FST for each rule. The actual disambiguator program that applies the
transducers to text we implemented ourselves. It reads the morphologically
analysed input in the Apertium stream format, converts it into the format
expected by \emph{fomacg}, applies the transducers to it, and then converts the
result back to the stream format. To emphasize its similarity to \emph{cg-proc},
VISL CG's rule applier, we named our program \emph{fomacg-proc}.

%\section{Bridging the Gaps}
%\label{sec:bridging}
%% TODO: what Mans did not cover in his paper (IFF, delimiters)?

\section{Methodology}
\label{sec:methodology}

Apertium includes Constraint Grammars for several
languages\footnote{http://wiki.apertium.org/wiki/Constraint\_grammar}. While
most of these are proper grammars, and are being actually used for morphological
disambiguation in Apertium, they are also too big and complex to be easily used
for the early stages of parser development. Therefore, we have written a small
Hungarian CG, aimed to fully disambiguate a short Hungarian story, which was used
as the development corpus. Since Hungarian is not properly supported by Apertium
yet, morphological analysis was carried out by Hunmorph~\cite{Tron:2005}, and
the tags were translated to the Apertium tagset with a transducer in \texttt{foma}.

The performance of \texttt{fomacg-proc} has been measured against that of
VISL CG. The programs were benchmarked with three Apertium CG grammars: the toy
Hungarian grammar mentioned earlier, the Breton grammar from the \texttt{br-fr}
language pair~\cite{Tyers:2010} and the version of the Finnish grammar originally
due to Karlsson in the \texttt{sme-fin} pair. % TODO: what is sme? Citations!
Seeing that in the early phases, only the Hungarian grammar was used for
development, results for the other two languages are reported only for the later
steps.

Each grammar was run on a test corpus. For Breton, we used the corpus
in the \texttt{br-fr} language pair, which consists of 1,161 sentences. There
are no Finnish and Hungarian corpora in Apertium; for the former, we used a
1,620-sentence excerpt from the 2013-Nov-14 snapshot of the Finnish Wikipedia,
while for the latter, the short test corpus used for grammar development. Since
the latter contains a mere 11 sentences, it was repeated 32 times to produce
a corpus similar in size to the other two. The Breton and Finnish corpora
were tagged by Apertium's morphological analyzer tools.

Since VISL CG implements CG-3, and \emph{fomacg} only supports CG-2, a one-to-one
comparison with the grammars above was infeasible. Therefore, we extracted the
subset of rules from each that compiled under \emph{fomacg}, and carried out the tests
on these subsets. Table~\ref{tab:grammar_size} shows the number of rules in the
original and the CG-2 grammars.

\begin{table*}[h]
  \centering
  \caption{Grammar sizes with the running time and binary size of the respective
           VISL-GC grammars}
  \label{tab:grammar_size}
  \begin{tabular}{ | l | r | r | r | r | }
  \hline
  \textbf{Language} & \textbf{Rules} & \textbf{CG-2 rules} &
  \textbf{Binary} & \textbf{Time} \\
  \hline
  Hungarian &   33 &   33 &   8kB / 16kB & 0.284s / 0.15s \\
  Breton    &  251 &  226 &  36kB / 44kB & 0.77s / 0.425s \\
  Finnish   & 1207 & 1172 & 184kB / 164kB & 1.78s / 0.74s \\
  \hline
  \end{tabular}
\end{table*}

We recorded both initialization and rule application time for the two programs,
via instrumentation in case of \emph{fomacg-proc} and by running the grammar
first on an empty file and then on the test corpus in case of \emph{cg-proc}.
However, as initialization is a one-time cost, in the following we are mainly
concerned with the time required for applying rules. The tests were conducted on
a consumer-grade laptop with a 2.2GHz Core2Duo CPU and 4GB RAM, running Linux.

%TODO: delete the While's, However's, etc.
\section{Performance Optimizations}
\label{sec:speed}

% TODO: this to the second section
Our implementation, much like that of \emph{fomacg} (and indeed, all recent
work on finite state CG) is based on the \emph{foma} library. We started out
with a naïve implementation that used solely stock \emph{foma} functions. Most
of the improvements below stem from the fact that we have replaced these
functions with custom versions that run much faster. The final implementation
abandons \emph{foma} entirely, but for the data structures. In the future, we
plan to discard those as well, making our code self-contained.

The program loads the transducers produced by \emph{fomacg} and applies them
to the text. The input is in the Apertium stream
format\footnote{http://wiki.apertium.org/wiki/Apertium\_stream\_format}
and it is read cohort-by-cohort. A \emph{foma} FST is used to convert each
cohort to the format expected by the rule transducers, and to convert the final
result back.

To tokenize the text to sentences, we modified \emph{fomacg} to compile the
\emph{delimiters} set and emit it as the first FSA in the binary representation
of the grammar. \emph{fomacg-proc} reads the input until a cohort matches this
set and then sends the accumulated sentence to the rule applier engine.

The rules are tested one-by-one, section-by-section, to see if any of them can
be applied to the text. Once such a rule is found, the associated FST is
executed on the text. As it is possible that a rule that was not applicable to
the original text would now run on the modified one, testing is restarted from the
first section after each rule application. The process ends when no more
applicable rules are found.

% TODO Two subsections: FST & CG-specific optimizations (?)
\subsection{Naïve Implementation}
\label{sec:speed_naive}

The first version of the program used the \texttt{apply\_down()} \emph{foma}
function both for rule application and format conversion. As \emph{fomacg}
generated a single FST for a rule, rule testing and execution was done in the
same step, by applying the FST. Whether the rule was actually applied or not was
decided by comparing the original sentence to the one returned by the function.

The first row in Table~\ref{tab:evaluation} shows the running time for the
Hungarian grammar. At 6.4s, the naïve implementation runs more than 20 times
slower than VISL-GC (see Table~\ref{tab:grammar_size}). Luckily a far cry from
the 1,500 reported by Peltonen, but clearly too slow to be of practical use.

\subsection{FST Composition}
\label{sec:speed_composition}

Another way to apply a rule is to convert the input sentence into a
single-path FSA with the same alphabet as the rules and compose the rule FST
on top of it. To check if the rule has actually be applied, the input
automaton was intersected with the result. Unfortunately, this method proved to
be much slower than the application-based one; composition alone took 28.3
seconds on our corpus, while the intersection pushed it up to 45s. Therefore we
decided to abandon this path altogether.

\subsection{Deletion of Discarded Readings}
\label{sec:speed_deletex}

The original transducers replace the \texttt{\#0\#} in front of discarded readings
with \texttt{\#X\#}. Our first optimization comes from the observation that
deleting these readings instead would not make the transducers any more complex,
but would shorten the resulting sentence, making subsequent tests faster.
Moreover, it allows the engine to recognize actual rule application by simply
testing the length of the output to the input sentence, an operation slightly
faster than byte-for-byte comparison.

Table~\ref{tab:evaluation} reports an approximately 8\% improvement. While not
self-evident, this benefit remained in effect after our subsequent optimizations.

\subsection{FSA-based Rule Testing}
\label{sec:speed_fsa}

Theoretically, further speed-ups could be achieved by separating rule testing
and application, using finite state automata for the former. Automata are faster
than transducers for two reasons: first, there is no need to assemble an output;
and second, a FSA can be determinized and minimized, while \emph{foma} can
only make a FST deterministic by treating it as a FSA with an alphabet of the
original input:output pairs, which does not entail determinism in the input.

As the third row in table~\ref{tab:evaluation} shows, the idea does not immediately
translate well to practice. The fault lies with the \texttt{apply\_down()}
function, which, being the only method of running a finite state machine in
\emph{foma}, was designed to support all features of the library. It treats
automata as identity transducers, and fails to capitalize
on the aforementioned advantages of the former. To realize the benefits of
FSA-based testing then, a custom function is required.

\subsection{Custom FSA/FST Application}
\label{sec:speed_custom}

\texttt{apply\_down()} supports the following features~\cite{Hulden:2009b}:
\begin{itemize}
  \item Conversion of the text to symbols (single- and multi-character)
  \item Regular transitions and flag diacritics
  \item Three types of search in the transition matrix (linear, binary and indexed)
  \item Deterministic and non-deterministic operation
  \item Iterators (multiple invocations iterate the non-deterministic outputs)
\end{itemize}

Our use-case makes most of these features superfluous. \emph{fomacg} uses
multi-character symbols, but not flag diacritics. To maximize the performance
gains, the rule testing automata must be minimal (hence deterministic), so
there was no need for non-determinism and iterators. Finally, by modifying
\emph{fomacg} to sort the edges of all grammar machines, we could ensure that
binary transition search alone suffices.
% TODO: explain that indexed is not much better?

The custom FSA applier function that implements only the necessary features
was employed for both rule testing and finding the delimiter cohort. As a result,
running time went down to 1.45 seconds (see table~\ref{tab:evaluation}), a 75\%
improvement.

A similar function was written for input-deterministic minimal transducers.
While not applicable to the non-deterministic rule FSTs, it could
replace \texttt{apply\_down()} for the conversion between the Apertium and  % TODO: eh
the \emph{fomacg} formats, further reducing the running time to 1.275 seconds.

% TODO: rule runner to here, too? Enumarate where we use FSM's?
% TODO TODO TODO BIMACHINE!!!!!

% TODO: br
What we can take home from the last two sections is that when speed is paramount,
relying blindly on generic libraries may not only lead to suboptimal performance,
but may also produce counterintuitive results.
Conversely, libraries may benefit from including specialized implementations
for different use-cases. For example, \emph{foma} has all information at hand
to decide if a FST is deterministic, whether it supports binary search or not,
etc. and so, providing specialized functions (even private ones hidden behind
\texttt{apply\_down()}) would improve its performance substantially in certain
situations.
% TODO TODO All configurations in the list above could have specialized functions.

\subsection{Exploiting CG Structure}  % TODO: ?
\label{sec:speed_cg}

In this chapter, we review the improvements made available by the
characteristics of our CG representation. The first of these is functionality:
even though the rule FSTs are non-deterministic, the input-output mapping is
one-to-one~\cite{Hulden:2011}. It was thus possible to implement the
non-deterministic version of the FST runner function described in the last
section without the need for an iterator feature, and to use it for rule
application. The last usage of the generic \texttt{apply\_down()} function  % TODO: usage
thus eradicated, the running time dropped to 1.05 seconds (see
table~\ref{tab:evaluation}).

% TODO: is this true for SFST and OpenFST? cite!
Internally \emph{foma}, similarly to other FST toolkits, represents elements
of the $\Sigma$ alphabet as integers. The conversion of text into tokens in
$\Sigma$ is a step usually taken for granted in the literature, but it
contributes to the execution time of an FST to a significant extent. In
\emph{foma}, token matching is performed by a trie built from the symbols in
the automaton's alphabet. Our custom DFSA runner function (see
section~\ref{sec:speed_custom}) spends about 60\% of its time applying this trie.

The two enhancements below have helped to all but negate the cost of token
conversion. The first of these exploits the fact that in the \texttt{fomacg}
format, symbols are separated by space characters. Instead of passing the
input string to each FSM, we split it along the spaces, and pass the resulting
string vector to the machines. This is a rather small change, and while the
Hungarian grammar benefited almost nothing, the running time of the Breton
grammar improved by 40\%.

The second enhancement came from the observation that all rule testing automata
and rule transducers accept the same CG tags. It is thus possible to generate an
automaton whose alphabet is the union of those of the other machines. This
automaton could be used to convert the input sentence into a vector of $\Sigma$
ids, and then this vector could be sent to the other machines, relinquishing the
need of repeated conversions.

Both \emph{fomacg} and \emph{fomacg-proc} had to be modified to account for the
changes. The former now creates the converter FSA and saves it as the
second machine in the binary grammar file. Also, since the ids that correspond
to a symbol are unique to each machine, we added a postprocessing phase that
replaces the ids with the "canonical" ones in the converter FSA. 
\emph{fomacg-proc} then converts the input to ids using the converter
automaton's trie, and sends the vector to the rule machines. The rule machines
treat the vector as their input, with a caveat: ids not in the alphabet of the
machine in question are replaced by \texttt{@IDENTITY\_SYMBOL@}, so that they
are handled in the same way as before.

Table~\ref{tab:evaluation} shows that factoring the symbol conversion out from
the individual machines resulted in huge savings: the running time of the
Hungarian setup improved by 70\% to 0.32 second; the Breton one by 40\% to 1.55
second.

\begin{table}[h]
  \centering
  \caption{Effects of the optimizations on running time}
  \label{tab:evaluation}
  \begin{tabular}{ | l | r | r | }
  \hline
  \textbf{Version} & \textbf{Hungarian} & \textbf{Breton} \\
  \hline
  Naïve (\ref{sec:speed_naive})                 & 6.4s   & -- \\
  Composition (\ref{sec:speed_composition})     & 6.4s   & -- \\
  Delete readings (\ref{sec:speed_deletex})     & 5.9s   & -- \\
  FSA rule testing (\ref{sec:speed_fsa})        & 10s    & -- \\
  Custom FSA runner (\ref{sec:speed_custom})    & 1.45s  & -- \\
  Custom format-FST (\ref{sec:speed_custom})    & 1.275s & 6.8s \\
  Input partitioning (\ref{sec:speed_cg})       & 1.15s  & 4s \\
  Custom rule applier (\ref{sec:speed_cg})      & 1.05s  & 2.6s \\
  One-time conversion (\ref{sec:speed_cg})      & 0.32s  & 1.55s \\
  \hline
  \end{tabular}
\end{table}

\section{Complexity Analysis}
\label{sec:complex}

Tapanainen~\cite{Tapanainen:1999} proves that the worst-case time complexity
for disambiguating a sentence in his CG-2 parser is $\mathcal{O}(n^3k^2G)$,
where $n$ is the length of the sentence, $k$ is the maximum number of readings
per word, and the grammar consists of $G$ rules. The explanation is as follows:
testing a cohort with a single rule can be done in $\mathcal{O}(nk)$; 
the whole sentence in $\mathcal{O}(n^2k)$. This process must be repeated for
each rule, yielding $\mathcal{O}(n^2kG)$. Finally, in the worst case, a rule
only removes a single reading, so it takes $n(k - 1)$ rounds to disambiguate the
sentence, resulting in the aforementioned bound.

Hulden~\cite{Hulden:2011} showed that if the rules are compiled to bimachines,
the complexity decreases to $\mathcal{O}(n^2k^2G)$, due to the fact that a
bimachine can be applied to a sentence of $nk$ tokens in $\mathcal{O}(nk)$
(linear) time, instead of the $\mathcal{O}(n^2k)$ suggested by Tapanainen.
However, \emph{fomacg} does not include bimachine factorization as of yet.

While this work has left the theoretical limit untouched thus far, it improved
on three aspects of the complexity.
First, unlike \emph{foma}, our specialized FST application functions can take
advantage of the properties of automata and bimachines, and actually run them
in $\mathcal{O}(nk)$ time.
Second, the constant in the $\mathcal{O}$ has been decreased as a result of
extensive optimization.
Third, rule testing automata have been introduced which, being minimal,
can also be applied in $\mathcal{O}(nk)$ time. Assume that in a round $G_a$
rules can be applied to the sentence and $G_u$ cannot, $G_a + G_u = G$. With
minimal automata for rule testing the round finishes with $2G_a + G_u$ machine
applications, instead of the $2G$ required by bimachines. The facts that usually
$G_a << G$ and that automata can be applied faster than transducers result in a
performance improvement over the pure bimachine setup.

% TODO: Anssi
% TODO: SuperTagger -> prefiltering Finite State to the Rescue (Extended Finite
%       State Models of Language)

\subsection{Beyond the $\mathcal{O}(n^2k^2G)$ Bound}
\label{sec:complex_idea}

Here we present an idea, due to Mans Hulden, that allows the system to
theoretically overcome the $\mathcal{O}(n^2k^2G)$ average complexity bound.
This section describes the method, and investigates its feasibility; the
next section contains the evaluation.

The idea is based on the fact that regular languages are closed under the union
operator. If there are two automata, $FSA_{G_a}$ and $FSA_{G_b}$ that test the
rules $G_a$ and $G_b$, respectively, then it follows that their union,
$FSA_{G_{ab}}$, accepts a sentence iff either $G_a$ or $G_b$ is applicable to it.
If $FSA_{G_{ab}}$ is minimized, it runs in $\mathcal{O}(nk)$ time, the same as
$FSA_{G_a}$ and $FSA_{G_b}$.

% TODO: allows us -> enables decreasing?
The union FSA allows us to implement hierarchical rule checking. In this example,
testing if any of the two rules match a sentence with only the original
automata requires a check with both. Instead, we can apply $FSA_{G_{ab}}$ first.
If neither rule is applicable, the automaton will not accept the sentence, and
no further testing is required. If one of the rules is, $FSA_{G_a}$ (or
equivalently, $FSA_{G_b}$) must be run against the sentence to see which. In
practice, if we pick two rules from a CG in random, we shall find that the
majority of the sentences will not match either, hence the number of tests may
be reduced substantially.

There is no need to stop here: we can take two union automata, and merge them
again. It is easy to see that if we represent the rule testing automata in a
graph, where a node is a FSA, and two nodes are connected iff one was created
from the other via union, then we get a binary tree. For a grammar of $G$ rules,
a binary tree of $logG$ levels can be built. Such a tree can confirm with a
single test if a sentence does not match any of the rules, or find the matching
rule in $logG + 1$ tests, if one does. Accordingly, in theory this method
allows us to improve the average- and best-case complexity bounds of the system
to $\mathcal{O}(n^2k^2logG)$ and $\mathcal{O}(nk)$, respectively.  % TODO: rewrite this
(Clearly, for grammars with several sections, instead of a single tree that
contains all rules, one tree must be built for each section to preserve rule
priorities. However, this does not affect the reasoning above.)  % TODO: reasoning?

The bottleneck in this method is memory consumption. The size of the FSA
resulting from a non-deterministic union operation is simply the sum of the
sizes of the original automata. To achieve the speed-up described above, however,
the rule checking automata must be determinized, which may blow up their size
exponentially. Therefore, building a single tree from all rules is infeasible.
A compromise solution is to construct a forest of 2--4 level trees, which still
fits into the memory and provides similar benefits to a single tree, though to
a smaller extent.

\subsection{Evaluation}

The forest can be assembled in several ways; we experimented with two simple
algorithms. Both take as input a list of rule testing automata, which are
encapsulated into single-node trees. Before each step, the trees are sorted by
the size of the automata in their roots.

The first algorithm, \textit{SmallestFirst}, unifies the two smallest trees in
each step, until the root FSA in each tree is above a size limit (1000 states in
our case).
The second, \textit{FixedLevel}, aims to create full, balanced binary trees:
in a single step, it unifies the smallest tree with the largest, the second
smallest with the second largest, etc, and repeats the process until the trees
reach a predefined height.

% TODO: it can be seen?
Table~\ref{tab:trees} lists the running times and memory requirements of the
resulting forests. It can be seen that hierarchical rule testing
indeed improves performance: even a single level of merging results in 30-42\%
speedup. However, it is also immediately evident that aside from special cases,
the disadvantages overweight the benefits: memory usage and binary size grow
exponentially, affecting compilation and grammar loading time as well, and very
soon we run into the limits of physical memory. This has prompted us to give up
on hierarchical rule testing.

\begin{table*}[h]
  \centering
  \caption{Performance and storage requirements of rule testing trees}
  % TODO: find a way to add * and +
  %\caption{Performance and storage requirements of rule testing trees\\
  %         \rtextsuperscript{*}State count limit was 500\rhspace{3em}
  %         \rtextsuperscript{+}Did not fit into memory}
  \label{tab:trees}
  \begin{tabular}{ | l | r | r | r | r | r | }
  \hline
  \textbf{Language} & \textbf{Algorithm} & \textbf{Initialization} &
  \textbf{Disambiguation} & \textbf{Memory} & \textbf{File Size} \\
  \hline
  Hungarian & (flat) & 0.028s / 0.16s & 0.32s / 0.18s & 0.5\% & 60kB \\
  Hungarian & FixedLevel(3) & 0.77s / 0.33s & 0.235s / 0.13s & 2.1\% & 7.1MB \\
  Hungarian & Smallest First & 0.62s / 0.29s & 0.234s / 0.13s & 1.9\% & 5.9MB \\
  Breton    & (flat) & 0.5s / 0.227s & 1.55s / 0.83s & 5.1\% & 1.5MB \\
  Breton    & FixedLevel(2) & 1.8s / 0.82s & 1.09s / 0.573s & 9.6\% & 7.4MB \\
  Breton    & Smallest First & 11.14s / 4.78s & 1.05s / 0.57s & 28.7\% & 60MB \\
  Finnish   & (flat) & 1.5s / 0.6s & 22.87s / 9.3s & 21.8\% & 7.2MB \\
  Finnish   & FixedLevel(2) & 3.64s / 1.6s & 13.28s / 5.25s & 32.3\% & 28MB \\
  Finnish   & SmallestFirst\textsuperscript{*} & 20.75s / 9.4s & 9.95s / 4s & --\textsuperscript{+} & 198MB \\
  \hline
  \end{tabular}
\end{table*}

\section{Memory Savings}
\label{sec:memory}

\subsection{Symbol Conversion}
\label{sec:sub_token}

The use of a single converter automaton has not only resulted in improved
performance, but it has also opened a way to decrease the storage space
requirements of the grammar as well. The trie that convert the 
machine's alphabet to integer ids in \emph{foma} takes up space; depending on
the number and length of the symbols in bytes, this trie may be responsible for
a considerable portion of the memory
footprint of an automaton. Given the number of rules in an average CG grammar,
it is easy to see how this trivial sub-task may affect the memory consumption of
the application, as well as the size of the grammar binary.
As the job of token matching has been delegated to the symbol
automaton (see section \ref{sec:speed_cg}), we no longer maintain
separate tries for all individual FSAs. 

Table \ref{tab:sigma_memory} presents the resulting memory savings.
We report numbers for the raw grammars (L1), as well as for two- and three-level
condition trees (L2-3). It is not surprising that the raw grammars see the
largest improvements; here the tries accounted for 70-80\% of the memory
usage. As the trees get higher, the number of states and edges grows more
rapidly than does the number of tries and the savings become more modest;
however, the approximately 30\% achieved for a three-level condition tree is
still substantial.  % TODO: what?

\begin{table}[h]
  \centering
  \caption{Improvements in memory usage due to removing the sigma trie. Memory
           consumption is measured as a percentage of the 4GB system memory}
  \label{tab:sigma_memory}
  \begin{tabular}{ | l | r | r | r | }
  \hline
  \textbf{Language} & \textbf{Before} & \textbf{After} & \textbf{Reduction} \\
  \hline
  hun L1 & 0.5\% & 0.1\% & 80\% \\
  hun L3 & 2.1\% & 1.5\% & 28.57\% \\
  br L1 & 5.1\% & 1.3\% & 74.5\% \\
  br L2 & 9.6\% & 4.4\% & 54.16\% \\
  fin L1 & 21\% & 4.1\% & 80.47\% \\
  fin L2 & 32.3\% & 8.9\% & 72.44\% \\
  \hline
  \end{tabular}
\end{table}

\subsection{Economizing Condition Trees}

% TODO lots of -> ???
As already mentioned in section \ref{sec:complex_idea}, while conditions trees speed up
rule matching, they also take up lots of space. The main reason is, as we have
seen, the pace with which the size of the merged rule testing automata grows.
There is, however, another factor at work here: each tree contains a
FSA in each of its nodes. For a tree of $L$ levels, this translates to $2^L - 1$
automata. Of course, the automata lower in the tree are much -- possibly
exponentially -- smaller than the ones at the top, but in practice their
contribution is not negligible.

On closer inspection, however, we can see that it is not necessary to have a
condition check in each node. If we are at a non-leaf node $N$ with children
$L$ and $R$ that matches the condition, it is enough to check if the FSA in
$L$ accepts the sentence; if not, we can automatically assume that $R$ does.
Hence, we removed the condition automata in the right branches of all non-leaf
nodes, thereby decreasing the number of condition FSAs to $2^{L-1}$. To maximize
the effect, if the condition FSA on the left branch was bigger than the one on
the right, the branches were swapped.

Table \ref{tab:XXX} shows the results.
% TODO: apparently I haven't implemented this yet! :)

\subsection{FSA Compression}

Although the memory saving techniques described in the last two sections help to
keep the memory consumption of the grammar in control, the potentially
exponential increase in size of the merged rule condition FSAs still prohibits
the building of arbitrary large trees, and thereby prevents us from reaching
the theoretical bound $\mathcal{O}(n^2k^2logG)$. The foma library already
implements some storage compression methods, such as representing the transition
matrix in row-indexed form, similarly to the method described in Kiraz~\cite{Kiraz:2001}.
Therefore, to make the idea in section~\ref{sec:complex_idea} feasible, an algorithm that
can compress FSAs aggressively is required. % TODO word order
In this section, we survey the literature in order to find such a method.
% TODO: compute the sparsity of our FSAs? (Kiraz)

Automata Mista, due to Huet~\cite{Huet:2003,Huet:2005}, is an applicative
method for describing finite state machines. It represents the FSA or FST as
a forest of tries, each corresponding to a deterministic subset of the paths in
the state space. Non-deterministic transitions and loops are handled by
"jumping" to an arbitrary node in the forest, which is located via a global or a
local addressing scheme. The trie format supports compression in two ways: it
facilitates path sharing and the compact representation can further shrink the
state space by merging non-branching paths. This technique managed to decrease
the size of a Sanskrit lexicon by 87.1\%~\cite{Huet:2005}.

Drobac et al.~\cite{Drobac:2014} present an algorithm for lossless
hyperminimisation of lexical transducers using automatically induced flag
diacritics. The idea behind the algorithm is to replace rightward continuations
with setting the corresponding flag. The authors report savings between
12.6--96.6\%.

% TODO: analyzer or analyser? minimi[sz]ation?
The above methods have similar characteristics that prevent us from applying them to
our case. First, both are aimed at optimizing mostly deterministic, loop-free
automata, such as those created for morphological analyzers. Second, both rely
on techniques that are beyond the power of the simple automaton interpreter
we are using (tries for Automata Mista and flag diacritics for XXX), and which
may incur significant runtime costs; indeed, the hyperminimized transducers run
12-200 times slower than the original ones~\cite{Drobac:2014}. 

A task-specific framework, one based on inward deterministic automata has
been proposed for CG parsing~\cite{Yli-Jyra:2011}. The paper reports a binary
size similar to the original grammar size. However, as the framework breaks away
from the practice of direct rule application followed in this paper and
in~\cite{Hulden:2011,Peltonen:2011}, closer inspection remains for future work. 

% TODO: read
%       Reducing memory requirements in reachability-based finite automata operations (http://onlinelibrary.wiley.com/doi/10.1002/spe.561/abstract)
%       On the state complexity of semi-quantum finite automata (http://arxiv.org/abs/1307.2499)
%       State Complexity Research and Approximation (http://link.springer.com/chapter/10.1007/978-3-642-22321-1_5)
%       State Complexity: Recent Results and Open Problems (http://iospress.metapress.com/content/veemqfxm6n48cpxn/)
%       Experiments with Automata Compression (http://link.springer.com/chapter/10.1007/3-540-44674-5_8)

\section{Conclusions}
\label{sec:conclusion}
We set out with the goal of creating a fast Constraint Grammar parser based on
finite-state technology. Our aim was to achieve better performance on the task
of morphological disambiguation than the current state-of-the-art parser
VISL CG-3. We used the CG grammars available in the Apertium machine translation
project.

Our goals were partially fulfilled: while the speed of our parser falls short
of that of VISL CG-3, we have made advances on the state-of-the-art
free/open-source FST implementations of CG. We based our system on the
\emph{fomacg} compiler, and extended it in several ways. Our parser  % TODO bimachines?
uses optimized FST application methods instead of the generic \emph{foma} variant
used by previous implementations, thereby achieving better performance.
Further optimizations, both memory and runtime, were realized
by exploiting the properties of FSTs generated from a CG. We report real-world
performance measurements with and without these optimizations, so their efficacy
can be accurately evaluated.
A new method for rule testing has also been proposed, which in theory is capable
of reducing the worst-case complexity bound of CG application to
$\mathcal{O}n^2k^2logG$. Unfortunately, the method proved to be infeasible in
practice.

Our main finding is that implementation matters: a too generic FST library
hinders performance and can even make a theoretically faster algorithm slower
in practice. Using bimachines and rule testing automata should have speeded up
rule application, but did so only after we implemented our own, specialized
FST functions. Since \emph{foma} has all necessary information about an FST in
place to decide the right application method, incorporating our functions into
it, or other FST libraries, could benefit applications beyond the scope of CG.

% TODO: Read and add Yli-Jyra:2011 -- a different method (I also was thinking
%       about a two-step method, where the first level converts cohorts to 
%       some representation (the sets it contains, etc.), and the second level
%       does something else).

\section*{Acknowledgements}
This research was conducted under the Google Summer of Code 2013 project
Rule-based finite-state disambiguation\footnote{https://google-melange.appspot.com/gsoc/project/google
/gsoc2013/davidnemeskey/14002}. The authors would like to thank Mans Hulden for
his insights.

% TODO: citations:
\cite{Karlsson:1990} % Karlsson:1990: the Finnish grammar
\cite{Hulden:2011} % Hulden:2011: Mans's paper
\cite{Peltonen:2011} % Peltonen:2011: the other try... should we include it at all?
\cite{Tapanainen:1996} % Tapanainen:1996: CG-2

\bibliographystyle{acl}
\bibliography{apertium_cg}
\end{document}
